\section{Issues with A.I in healthcare}

\endquote{there are few controversies such as increased chances of data breaches, concern for clinical implementation, and potential healthcare dilemmas}. {Rahman MA, Victoros E, Ernest J, Davis R, Shanjana Y, Islam MR. Impact of Artificial Intelligence (AI) Technology in Healthcare Sector: A Critical Evaluation of Both Sides of the Coin. Clin Pathol. 2024 Jan 22;17:2632010X241226887. doi: 10.1177/2632010X241226887. PMID: 38264676; PMCID: PMC10804900.}


\section{Evolution of A.I}
\enquote{The first of many prototypes of how AI in medicine could positively impact the future started in the late 70s with the introduction of the consultation program Causal-Associational Network (CASNET). {2} The program could use disease data, apply it to an individual, and give advice to the physician on how to help the patient manage the disease. {2} Later, a bacterial pathogen and antibiotic treatment diagnostic AI developed from MYCIN to EMYCIN to INTERNIST-1 in just a few years. {2} The evolution of this system elaborated on the already extensive AI medical knowledge to help assist primary care physicians (PCPs). {2} The AI program that prompted the most influence of AI in medicine was introduced in 1986-DXplain. {2} PCPs were able to input their patient’s symptoms, and the program responded with a diagnosis, along with a description of the disease and additional references for the physicians. {2} The program started with 500 diseases and has now expanded to over 2400. {2} The early 2000s introduced Watson, an open-domain question-answering system. {2} This system used the electronic medical record with other electronic resources to provide physicians with evidence-based solutions to their patients’ questions. {2} Watson was later expanded upon for the exploration of medical research into new areas. Thus, beginning the expansion of AI into other areas such as pharmacy and patient intake at primary care practices. {2}}

2 - { Kaul V, Enslin S, Gross SA. History of artificial intelligence in medicine. Gastrointest Endosc. 2020;92:807-812.}