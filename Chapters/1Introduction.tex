%************************************************
\chapter{Einleitung}\label{ch:introduction}
%************************************************
\section{Gegenstand}
In the constantly and rapidly changing healthcare sector, the effective management and efficient acquisition of knowledge is essential. Health information includes all data relating to a person's health or the accurate provision of healthcare services. This includes clinical and outpatient records, diagnostic images, laboratory results and administrative details. This information is relevant to a variety of stakeholders, including healthcare professionals, administrators, patients and researchers. The continuous flow of health and professional information from various sources is then the basis for subsequent diagnosis and treatment. As digitalization progresses, new technologies are being introduced that enable the continuous flow of health and professional information from various sources. 
The goal of Health informatics is to study and pursue "[...]the effetive uses of biomedical data, information, and knowledge for scientific inquiry, problem-solving, decision making, motivated by efforts to improve human health"1. 
The interdependence of health information and medical informatics in today's world highlights the importance of striving to progress and adapt to current demands in collaboration with modern technologies.

\section{Problemstellung}
As the amount of data increases, the need of efficient extraction, analyis and accessibiltiy by constantly developing information systems and technologies gains further relevance. It further shows the gaps that currently exist in efficient use of health information to make better decisions, improve research and achieve better healthcare outcomes. Machine learning algorithms embeded and scaled into Large Language Models (LLMs), can analyze scientific data, suchs as books, to make health information more accessible and increase the efficiency of retrieving essential knowledge.

\section{Motivation}


\section{Zielsetzung}\label{sec:zielsetzung}


\section{Aufgabenstellung}



\section{Aufbau der Arbeit}
