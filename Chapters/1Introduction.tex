%************************************************
\chapter{Introduction}\label{ch:introduction}
%************************************************
\section{Subject Matter}\label{sec:subject matter}
In healthcare sector and educational landscape, the efficient acquisition of domain-specific information is essential. 
In educational contexts such as universities, the basic practical and theoretical approach of students and teachers to complex domains, for example, medical informatics is mostly developed through literature such as \citet{bb2}.
%
Several approaches already exist, as mentioned in \cref{sec:motivation}, to make the information accessible without having to manually read a textbook, such as \citet{bb2}.
This aspect is so relevant because the content is mainly to be repeatetly viewed and accessed in short time windows, in scenarios such as exam or academic debate preparation. 
%
The process of extracting information not only requires accessibility for every user and short response times, but also has correctness and completeness, due to the need for academic consistency, as its highest goal. 
Textbooks and academic literature are fragmented information sources, where relevant information is complex and interconnected throughout sections. 
Combining these fragments completely and correctly in context and therefore achieve these highest goals pose the decisive issues to overcome. 
%
Large context window \ac{llm}'s, as well as various improved \ac{rag}-systems are recent advancements of AI to analyse vast quantities of data.
They open up new possibilities for overcoming the challenge of recognising complex contextual relationships, presenting the extracted information in context and at the same time, provide accessibility for any stakeholder.

\section{Problem Statement}\label{sec:problem statement}
For students and practitioners in particular, practical application requires a deep understanding of the interrelationships and the applicability of the concepts presented to specific work environments.
The existing literature is frequently extensive and the retrieval and synthesis of complex domain-specific information, such as within the field of medical informatics, is often hindered by the fragmentation and dispersion of relevant information across multiple sections and chapters.
Identifying relevant knowledge for specific problems requires students and practitioners to read large parts of a book in order to fully grasp individual concepts and their relationships to other topics.
The scope of the literature sources and the fragmentation in the definition of technical terms make it difficult to acquire knowledge quickly, especially for students who want to understand basic concepts correctly.
%
In response to the previous need for manual source, recent resource-constrained approaches, as mentioned in \cref{sec:motivation}, towards the process of extracting and integrating relevant information to an adequate answer, could not constantly satisfy certain criteria for the provided demands. 
Firstly, they remain incomplete, inaccurate and often do not meet academic standards, particularly when the relevant interconnected information is widely spread throughout the source.
%
Particulary \citet{Paul_Keller} showed that the issue of inaccuracy exists but emphasised that It could be overcome by using state-of-the-art models and with that an increase of hardware resources. 
\citet{Paul_Keller}'s conclusion leads us to another problem.   
Despite significant advancements in artificial intelligence and information retrieval technologies, the practical implementation of these systems in resource-constrained environments, such as smaller educational institutions or even private learning settings, remains a considerable challenge. 
Limited availability of high-performance hardware necessitates a focus on cost-effective and especially accessible solutions for real-world educational and healthcare environments and the targeted stakeholders, students, teachers and researchers.
%
Key issues:
\begin{itemize}
    \item \textbf{Academic standards}:\\
    Previous approaches do not meet the demands of academic standards in their provided responses. 
    \item \textbf{Fast applicability in resource-limited environments}:\\ 
    Resource-constrained settings lack the high-end hardware and often the time to individually advance \ac{llm}s and \ac{rag} systems for retrieving interconnected information from scientific literature.
\end{itemize}
%
Addressing these challenges is vital for enhancing the accessibility and utility of domain-specific knowledge, ultimately supporting both learning and research activities. 

\section{Motivation}\label{sec:motivation}
By leveraging these technologies, it becomes possible to retrieve relevant information from fragmented sources, as well as recognise, combine and precisely return contextual connections provided from the source as factually correct information. 
The central challenge remains in detecting a retrieval model that not only addresses the issue of fragmented knowledge but also ensures accessibility and factual reliability to a degree, where academic standards can be met.\\ 
In this context, \ac{qa} systems provide a practical and measurable approach to evaluate such information retrieval. 
These systems enable targeted queries to be processed efficiently, offering insights into their applicability in educational scenarios, as well as private settings.
% 
Previous projects, such as the \ac{snik} ontology, have made substantial progress in structuring information for hospital information systems management and therefore reach a certain level of accessibility. 
The integration of natural language processing in the context of a \ac{bell} resulting in the \enquote{Question Answering on \ac{snik}} \citep{hannesbell, hannesbell_skill}, as well as a similar approach in \citet{snikquiz}, attempted to combine complex knowledge networks with user-friendly access through natural English language queries. \citet{arneba}, on the other hand, explored an inverse approach using the \ac{snik} system. Instead of focusing on retrieving specific information through queries, this work aimed to provide the automatic generation of relevant questions, enhancing the system's usability in educational and training contexts.\\
%
These methods enabled are more structured approach to the information from the given sources and further a basic \ac{qa} but are limited in the handling of complex queries and interconnected information, which is scattered over sections.
The work of \citet{Paul_Keller} has already addressed these issues, recognizing the challenges associated with managing information systems, especially the difficulty in translating theoretical information into practical applications and dealing with fragmented definitions in the literature, particularly in \citet{bb2}. 
In response to these challenges, the attempt was to solve the issue by fine tuning a pre-trained transformer in order to provide a efficient and reliable \ac{qa}-system built on \citet{bb2} as a literary information base.
\citet{Paul_Keller} demonstrated that while these \ac{llm}'s can provide more intuitive and comprehensible responses, they still fall short in terms of accuracy and demand high-level hardware to meet the initial requirements.\\ 
%
The results suggests the potential for improvement. 
Such improvement may be reached through either a better training technique and the use of large scale hardware to fully exploit the potential of \ac{llm}'s, or to expand the model with further technics in order to reduce the occurrence of incorrect answers.\\
%
However, feasibility should be the focus for the stakeholders described, as currently available models already have the potential to represent the required standards for the scenarios under consideration. 

\section{Objectives}\label{sec:objectives}
The following goals of this work are assigned to the problem shown in \cref{sec:problem statement}.
\begin{itemize}
  \item goal G1:\\ 
    Answering questions about healthcare information systems in natural language using state-of-the-art language models with the help of \citet{bb2} 
  \item goal G2:\\
   Solving a sample exam of the module \enquote{Architecture of Information Systems in Healthcare}\footnote{\raggedright{}a module of the Master's program in Medical Informatics at the University of Leipzig, which is based on \citet{bb}} with the help of state-of-the-art language models.\@
   The main goal is first to compare with previous approaches. 
   The setup created in the process does not primarily claim to be generally reproducible.
   \item goal G3:\\ 
    Evaluate and compare available state-of-the-art models by utilising the \ac{qa}-system used by \citet{Paul_Keller} in relation to academic standards and requirements.
\end{itemize}
\section{Aufgabenstellung}



\section{Aufbau der Arbeit}
