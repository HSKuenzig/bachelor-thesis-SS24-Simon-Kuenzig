%************************************************
\chapter{Introduction}\label{ch:introduction}
%************************************************
\section{Subject Matter}\label{sec:subject matter}
In the healthcare sector and educational landscape, the efficient acquisition of domain-specific information is essential. 
In educational contexts such as universities, the basic practical and theoretical approach of students and teachers to complex domains, for example, medical informatics, is mostly developed through the literature such as \citet{bb2}.
%
Several approaches already exist, as mentioned in \cref{sec:motivation}, to make the information accessible without having to manually read a textbook, such as \citet{bb2}.
This element holds significant relevance, as scenarios like exam or academic debate preparation primarily require reiterating and accessing the content within brief time-frames.
%
The process of extracting information not only requires accessibility for all users and short response times, but also factual accuracy and completeness to ensure that scientific and academic information standards are met, providing reliable and consistent information based on the source material.
 
Textbooks and academic literature are fragmented information sources, where the relevant information is complex and interconnected throughout the sections. 
Combining these fragments completely and correctly in context poses the decisive issues to overcome. 
%
Large context window \ac{llm}'s, as well as various improved \ac{rag}-systems, are recent advancements of AI to analyse vast quantities of data.
They open up new possibilities for overcoming the challenge of recognising complex contextual relationships, presenting the extracted data in context, and, at the same time, providing accessibility for any stakeholder.

\section{Problem Statement}\label{sec:problem statement}
For students and practitioners in particular, practical application requires a deep understanding of the interrelationships and the applicability of the concepts presented to specific work environments.
The existing literature is often extensive, and the fragmentation and dispersion of relevant information across multiple sections and chapters often hinders the retrieval and synthesis of complex domain-specific information, such as within the field of medical informatics.
Identifying relevant information for specific problems requires students and practitioners to read large parts of a book in order to fully grasp individual concepts and their relationships to other topics.
The scope of the literature sources and the fragmentation in the definition of technical terms make it difficult to acquire knowledge quickly, especially for students who want to understand basic concepts correctly.
%
In response to the need for manual source research in the past, new, mostly resource-limited approaches, as mentioned in \cref{sec:motivation}, were developed. 
These mainly solved partial problems, but always showed limitations, inaccuracies in their answers and incompleteness with regard to the requirements. 
%
Particularly, \citet{Paul_Keller} showed that the issue of inaccuracy exists but emphasised that it could be overcome by using state-of-the-art models and, with that, increasing the hardware resources. 
His conclusion leads us to another problem.   
Despite significant advancements in artificial intelligence and information retrieval technologies, the practical implementation of these systems in resource-constrained environments, such as smaller educational institutions or even private learning settings, remains a considerable challenge. 
Limited availability of high-performance hardware necessitates a focus on cost-effective and especially accessible solutions for real-world educational and healthcare environments and the targeted stakeholders, students, teachers and researchers.
%
Key problems:
\begin{itemize}
  \item \textbf{Problem 1:} Fragmentation of information in \citet{bb2}, still makes it difficult to extract coherent and relevant information.
  \item \textbf{Problem 2:} Resource constraints prevent students and teachers from effectively using advanced techniques and applying them on \citet{bb2}.
  \todo{Habe die Probleme nochmal angepasst. Sie passen nun mehr zum Rahmen der Arbeit}
\end{itemize}
%
Addressing these challenges is vital for enhancing the accessibility and utility of domain-specific knowledge, ultimately supporting both learning and research activities. 

\section{Motivation}\label{sec:motivation}
\todo{Also da bin ich skeptisch. LLMs sind doch ziemlich eindeutig ein sinnvolles Retrieval Model für den Zweck.
Ich dachte unsere key challenge ist, das mit limited resources hinzubekommen?
Bitte korrigiere mich, wenn ich da was übersehen habe.
Das mit den limited resources kommt mir hier noch viel zu wenig zur Geltung.}
The key challenge is to find a retrieval model that not only solves the problem of fragmented knowledge and scientific and factual reliability, but also ensures accessibility, as well as availability at any time and in any application scenario. 
%
Previous projects, such as the \ac{snik} ontology, were able to structure the knowledge from literature sources, for example \citet{bb2}, about the management of health information systems. 
\todo{Ich empfehle immer einen Satz pro Zeile maximal außer in todonotes.
Das minimiert git diffs und erhöht die Übersichtlichkeit.
Für mich ist es kaum lesbar, da ich kein auto wrapping aktiviert habe im Texteditor.
Außerdem hast du dann mehrere todonotes von mir hintereinander und weißt nicht, auf welchen Satz sie sich beziehen.}
\todo{Der Satz 'it resulted in...' klingt unglücklich, kannst du das noch besser formulieren?}
\todo{Die Sätze zu SNIK Quiz sind sehr verwirrend, es klingt als wäre SNIK Quiz und Arnes BA was unterschiedliches aber genau das war doch sein Thema.}
\todo{Achtung! Bitte nicht citep und citet verwechseln.}
The work of \citep{hannesbell, hannesbell_skill} in the context of a \ac{bell} made use of that structured knowledge base and aimed to further improve accessibility of \ac{snik} through the integration of natural language processing.    It resulted in the \enquote{Question Answering on \ac{snik}}. \citet{snikquiz} attempted a gamified access to the knowledge contained within \ac{snik}. As \citet{arneba}, on the other hand, explored an inverse approach with the automatic generation of relevant questions as output, enhancing the \ac{snik} system's usability in educational and training contexts.
These projects focused on providing knowledge more intuitively, applying a user-friendly layer by using natural language queries.
%
However, they are limited when processing complex queries and cross-linked information scattered across sections.
The work of \citet{Paul_Keller} has already addressed these issues. The challenges associated with managing health information systems, especially the difficulty in translating theoretical information into practical applications and dealing with fragmented definitions in the literature.
Although his work focuses particularly on \citet{bb2} instead of \ac{snik}.
%
In \citet{Paul_Keller}, it was shown that with training and fine-tuning an open source \ac{llm} was able to outperform its untrained version and could provide more comprehensible responses with improved accuracy in explicit and limited reference to \citet{bb2}.
At the same time, compared to the GPT-4 model during inference, the self-training approach in a hardware-limited environment is confronted with clear capability differences.
%
Improvements may be reached through the use of large-scale hardware to fully exploit the capabilities of larger open-source \ac{llm}-models with more parameters, or by further expanding advanced, commercial API-accessible models, such as GPT-4, with novel techniques to reduce the occurrence of incorrect answers.\\
%
All future concepts should focus on feasibility and accessibility, as users do not have the time, expertise, or infrastructure to train models on their own.

\section{Goals}\label{sec:goals}
The following goals of this work are assigned to the problems shown in \cref{sec:problem statement}.
\begin{itemize}
  \item goal G1:\\
    Answering questions about healthcare information systems in natural language using state-of-the-art language models with \citet{bb2} as the literature information source. 
  \item goal G2:\\
   Solving a sample exam of the module \enquote{Architecture of Information Systems in Healthcare}\footnote{\raggedright{}a module of the Master's program in Medical Informatics at the University of Leipzig, which is based on \citet{bb2}} with the help of state-of-the-art language models.
   The main goal is to compare with previous approaches.
   The setup created in the process does not primarily claim to be generally reproducible.
   \item goal G3:\\
   Set up a reproducible application scenario for the efficient use of selected state-of-the-art \ac{llm} for self-selected domain specific sources.
\end{itemize}
\section{Task}
\begin{itemize}
  \item Task for goal G1
        \begin{itemize}
          \item Task A1.1:\\Evaluate the answer options of exam questions according to the same criteria as in \citet{Paul_Keller}.
          Current state-of-the-art language models are to be compared. Within the scope of this work, no claim to completeness with regard to these models can be made. 
          \end{itemize}
 \item Task for goal G2
    \begin{itemize}
          \item Task A2.1: The language models selected in this thesis are evaluated with regard to their ability to answer questions correctly in comparison to each other and to the results of \citet{Paul_Keller}.\@
          This enables a statement to be made about the possible use of these models in an academic and scientific environment.
        \end{itemize}
  \item Task for goal G3
        \begin{itemize}
          \item Task A3.1:\\Investigate reproducible approaches for the efficient use of the selected models.
          \item Task A3.2:\\Evaluate the reproducible approaches for the efficient use of the selected models and, if necessary or more efficient, set up an \ac{api} and \ac{gui}.
        \end{itemize}
\end{itemize}



\section{Structure of the thesis}
