%************************************************
\chapter{Einleitung}\label{ch:introduction}
%************************************************
\section{Gegenstand}
In the constantly and rapidly changing healthcare sector, the effective management and efficient acquisition of knowledge is essential. Health information includes all data relating to a person's health or the accurate provision of healthcare services. This includes clinical and outpatient records, diagnostic images, laboratory results and administrative details.\\

These informations are relevant to a variety of stakeholders, including healthcare professionals, administrators, patients and researchers. The continuous flow of health and professional information from various sources is then the basis for subsequent diagnosis and treatment. As digitalization progresses, new technologies are being introduced that enable the continuous flow of health and professional information from various sources.\\
 
The goal of Health informatics is to study and pursue~\enquote{[$\dots$]the effetive uses of biomedical data, information, and knowledge for scientific inquiry, problem-solving, decision making, motivated by efforts to improve human health}\footnote{\raggedright{}Jen MY, Mechanic OJ, Teoli D. Informatics. [Updated 2023 Sep 4]. In: StatPearls [Internet]. Treasure Island (FL): StatPearls Publishing; 2024 Jan-. Available from: \url{https://www.ncbi.nlm.nih.gov/books/NBK470564/}}.In educational contexts such as universities, the basic practical and theoretical approach to medical informatics is is developed through literature such as \citet{bb}. The ontology \ac{snik} \citep{semantischesnetz}, following the \ac{snik} meta-ontology and part of the \ac{snik} project of the Institute for Medical Informatics, Statistics and Epidemiology\footnote{\raggedright{}Institut für Medizinische Informatik, Statistik und Epidemiologie.\ \url{https://www.imise.uni-leipzig.de/Institut} (visited on 9.3.2023).}. at the University of Leipzig, exists to structure technical terms and roles of information management in hospitals. The use of this network enables a systematic representation of roles, entities and functions of information management in hospitals, independent of the definition of the underlying literature sources. The interdependence of health information and medical informatics in today's world highlights the importance of striving to progress and adapt to current demands in collaboration with modern technologies. Various artificial intelligence (AI) models can have profound effects on future work processes in multiple domains.

\section{Problemstellung}
As the amount of data increases, the need of efficient extraction, analyis and accessibiltiy by constantly developing information systems and technologies gains further relevance. It further shows the gaps that currently exist in efficient use of health information to make better decisions, improve research and achieve better healthcare outcomes. Machine learning algorithms embeded and scaled into Large Language Models (LLMs), can analyze scientific data, suchs as books, to make health information more accessible and increase the efficiency of retrieving essential knowledge.

\section{Motivation}


\section{Zielsetzung}\label{sec:zielsetzung}


\section{Aufgabenstellung}



\section{Aufbau der Arbeit}
