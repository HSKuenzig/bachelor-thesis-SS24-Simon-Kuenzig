%************************************************
\chapter{Einleitung}\label{ch:introduction}
%************************************************
\section{Gegenstand}
In the constantly and rapidly changing healthcare sector, the effective management and efficient acquisition of knowledge is essential. 
Health information includes all data relating to a person's health or the accurate provision of healthcare services. This includes clinical and outpatient records, diagnostic images, laboratory results and administrative details.\\

These informations are relevant to a variety of stakeholders, including healthcare professionals, administrators, patients and researchers. 
The continuous flow of health and professional information from various sources is then the basis for subsequent diagnosis and treatment. 
As digitalization progresses, new technologies are being introduced that enable the continuous flow of health and professional information from various sources.\\
 
The goal of Health informatics is to study and pursue~\enquote{[$\dots$]the effective uses of biomedical data, information, and knowledge for scientific inquiry, problem-solving, decision making, motivated by efforts to improve human health}\footnote{\raggedright{}Jen MY, Mechanic OJ, Teoli D. Informatics. [Updated 2023 Sep 4]. In: StatPearls [Internet]. 
Treasure Island (FL): StatPearls Publishing; 2024 Jan-. Available from:\url{https://www.ncbi.nlm.nih.gov/books/NBK470564/}.} 
In educational contexts such as universities, the basic practical and theoretical approach to medical informatics is developed through literature such as \citet{bb}. 
The ontology \ac{snik} \citep{semantischesnetz}, following the \ac{snik} meta-ontology and part of the \ac{snik} project of the Institute for Medical Informatics, Statistics and Epidemiology\footnote{\raggedright{}Institut für Medizinische Informatik, Statistik und Epidemiologie.
\ \url{https://www.imise.uni-leipzig.de/Institut} (visited on 9.3.2023).} at the University of Leipzig, exists to structure technical terms and roles of information management in hospitals. 
The use of this network enables a systematic representation of roles, entities, and functions of information management in hospitals, independent of the definition of the underlying literature sources. 
The interdependence of health information and medical informatics in today's world highlights the importance of striving to progress and adapt to current demands in collaboration with modern technologies.//


Large language models, such as GPT-4, machine learning algorithms and deep learning are increasingly influencing our everyday interaction and communication and opening up a wide range of possible applications. 
The impact of artificial intelligence and the transformation of many economic, medical and social areas can already be seen. 
55 percent of organizations have adopted AI to varying degrees \citep{mckinsey_ai_2023}, suggesting increased automation for many businesses in the near future. 
The capacity of AI to analyse vast quantities of data and transform its discoveries into user-friendly visual formats can expedite the decision-making process and, for instance in a medical context,  enable more informed and accurate diagnoses.\citep{esteva_skin_cancer}

\section{Problem Statement}
As the amount of data increases, the need of efficient extraction, analyis and accessibiltiy by constantly developing information systems and technologies gains further relevance. 
It further shows the gaps that currently exist in efficient use of health information to make better decisions, improve research and achieve better healthcare outcomes. 
Machine learning algorithms embeded and scaled into Large Language Models (LLMs), can analyze scientific data, suchs as books, e.g., \citet{bb}, to make health information more accessible and increase the efficiency of retrieving essential knowledge. 
In order to be able to jointly discuss the future of the healthcare system and its connection to the protagonists involved, such as the economic and social sectors, definitions, concepts and interrelationships should be clearly and above all uniformly presented. 
Only on the basis of a linguistic and content-related consensus can standardised theoretical concepts be followed by standardised practical implementations. In order to then drive forward the necessary restructuring, adaptation and ultimately improvement of existing information systems. 
An efficient, reliable, and easily accessible retrieval model is essential to access the content of this literature in a flexible and user-friendly manner. This ensures that important connections between stakeholders, concepts, structures, and definitions are maintained and presented consistently. 
Definitions and explanations of technical terms or topic-related concepts are often fragmented over several sections and are not always available as a clear definition. An explanation within a literature source usually takes place in different sections and in the course of the respective chapters. If the context is to be recorded, it is usually necessary to record the entire chapter or several sections in order to obtain the required information and its relationship to other concepts.

\begin{itemize}
\item Knowledge Fragmentation: The fragmentation of definitions and explanations across multiple sections and chapters complicates the extraction of coherent and comprehensive information
\item Inefficient Data Extraction: Traditional extraction methods are inefficient, leading to delays in accessing critical health information

\end{itemize}
\section{Motivation}


\section{Zielsetzung}\label{sec:zielsetzung}


\section{Aufgabenstellung}



\section{Aufbau der Arbeit}
