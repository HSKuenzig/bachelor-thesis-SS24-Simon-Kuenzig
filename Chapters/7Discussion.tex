%*****************************************
\chapter{Discussion}\label{ch:discussion}
%*****************************************
\section{Potential Improvements for the RAG Pipeline}

Future work may investigate the optimisation of the \ac{rag} pipeline.
The implementation within this thesis was designed to be functional and effective for the primary goal of a comparative evaluation, rather than being exhaustively fine-tuned for peak performance. 
Consequently, several parameters, such as \enquote{top\_k} and \enquote{Chunk\_size}, were set to sensible, commonly used values without empirical \textit{A/B testing}.
Systematically and detailed testing of the configuration settings could significantly enhance the pipeline's efficiency and the quality of its output.
%
\subsection{Chunk Size}
Possible improvements start with the configuration of the embedding.
The size of the chunks should be adapted to the source.
An in-depth analysis of the complexity and style of the literature could help to define the ideal chunk size and the appropriate overlap.
Detailed literature with a high number of concrete, clearly formulated facts, data and knowledge per sentence may allow for a smaller chunk size and lower overlap.
On the other hand, a text with extensive wording and paraphrasing of complex content requires larger chunks and overlap in order not to lose the context.
%
\subsection{Top-K}
Further optimisation of \ac{rag} can be achieved by adjusting the parameter \enquote{top-k}, the number of document snippets retrieved during the retrieval process.
The choice of \textit{k} is a critical consideration.
Retrieving a larger number of extracts from the database may seem advantageous, as they provide more context to the \ac{llm}.
However, this also increases the likelihood that irrelevant or redundant information (\enquote{noise}) will be processed.
This \enquote{noise} can dilute the relevant context, possibly causing relevant details to be lost from focus and leading to a deterioration in the \textit{precision} and conciseness of the answer ultimately generated.
Experimental studies would vary the parameter value while analysing the quality of the generated output by the respective \ac{llm}.
Identifying the highest possible quality of output likely correlates with the inclusion of a \textit{k} and a \textit{threshold}.
%
\subsection{Cost}
Furthermore, practical considerations of cost and latency are directly tied to the number and size of the retrieved snippets. 
A larger \textit{k} value results in a longer prompt being sent to the generative model, which in turn leads to higher \ac{api} costs and increased response times.
For the target group of students and practitioners operating under resource constraints, these factors are of significant importance.

Future research should therefore aim to identify the optimal balance.
Industry practice often suggests that a top-k value between $5$ and $10$ snippets provides a good compromise for many applications. A higher value ($10\geq$ should probably only be considered when it is known that the required information is highly distributed throughout the knowledge base.
The selected \ac{llm} should also have robust capabilities in dealing with very long and complex contexts, which can now be nearly regarded as a given standard due to rapid advances in technology and skills.


To move beyond heuristics, a systematic investigation is recommended. This could involve conducting a series of A/B tests or a grid search, varying both the chunk size and the top-k value, and measuring the impact on the evaluation metrics, particularly the Macro-F1 score and the automated DeepEval scores for faithfulness and relevance. Such an empirical approach would determine the ideal configuration for this specific knowledge base and question set, providing a valuable blueprint for developing a highly optimised, rather than merely functional, QA system.


----
possibities to discuss:

additional ranking,deduplication, semantic expansion, post‑hoc re-ranking. --> future improvements

- Score ordering manipulations (e.g., rescoring, blending multiple queries).
- MMR 
- Vector quantisation / HNSW parameter tuning (M, $ef_construct, ef_search$).
- Sparse/dense hybrid (e.g., BM25 + vector fusion).
- Multi-vector / multimodal fields.
- Sharding or payload indexing optimisation (e.g., indexing frequently filtered - payload fields).
- Reranking with a cross-encoder after initial recall.
Time decay / recency weighting.
- Semantic attribute weighting (e.g., boosting certain metadata categories).
- Query expansion 
- Using Qdrant's native gRPC for lower latency (LangChain uses HTTP client).








